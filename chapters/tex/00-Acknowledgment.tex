First and foremost, I am indebted to my supervisor Solve Sæbø who picked
me up from nowhere and brought me into a scientific community by giving
me a chance to pursue this degree. His inspiration and encouragement
have been an essential element in the course of this journey. I am
grateful to my co-supervisor Trygve Almøy for being a mentor, a friend,
a colleague, and a guardian and guiding and supporting me throughout
this period. He has always been there for me with my frustration and
excitement.

I am forever grateful to my father Narayan Prasad Rimal and mother
Bhagawati Rimal for their continuous support and encouragement. Their
belief in me and push for my education have shined the light in my hard
and easy times. I am also thankful to my dear wife Junali Chhetri who
has inspired me every step of my life and help me to better understand
myself. And of course, a thank goes to my beloved son Nirvan Rimal who
has understood my busy time during this study.

I would also like to thank Professor Inge Helland for his insight,
suggestion and comment on many mathematical problems on various
statistical methods presented in the thesis.

Last, but importantly, my thank goes to the Biostatistics group with
whom I have collected beautiful memories. Thanks to all the members of
the group from past and present who have always made my stay at NMBU
happy, festive and full of joy.
