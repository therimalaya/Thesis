A linear regression model defines a linear relationship between two or
more random variables. The random variables that depend on other random
variables are often called response variables and the independent random
variables are called predictor variables. In most cases not all
variation is relevant for regression, i.e.~only a certain amount of the
variation in the predictors is relevant and only so for a part of the
variation in the response. This leads to a reduction of the linear
regression model where one can imagine a subspace of the space spanned
by the predictor variables that contains all the relevant information
for a subspace of the space spanned by the response variables.

In this thesis we attempt to compare some new methods which are based on
the envelope model and some established methods such as principal
components regression (PCR) and partial least squares regression (PLS).
The comparison tests these methods on their performance of producing
minimum prediction and estimation error while modelling data simulated
with specifically designed properties. For the simulation we have also
created an R-package called \texttt{simrel} with a web interface.

A simulation model for a multi-response multivariate linear model, on
which the simulation tool is based, is discussed in the first paper.
This paper prepares a basic foundation for the simulations with the
concept of reduction of regression models. The second paper discusses
the similarities of the envelope, PCR and PLS population models. This
paper compares the prediction performance of several multivariate
methods using a model with a single response.

The final two papers make an extensive investigation evaluating the
prediction and estimation performance of established (PCR, PLS1 and
PLS2) and newly developed envelope based (Xenv and Senv) methods.
Unsurprisingly the study found that not one method dominates in all
situations, but their performance depend on the properties of the data
they model. However, the envelope based methods have shown remarkable
performance in many cases, both in prediction and estimation. The study
also recommend researchers to use and evaluate the envelope methods.
