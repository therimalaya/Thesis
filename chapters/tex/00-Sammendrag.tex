En lineær regresjonsmodell definerer et lineært forhold mellom to eller
flere tilfeldige variabler. De tilfeldige variablene som er avhengige av
andre tilfeldige variabler, kalles ofte responsvariabler, og de
uavhengige tilfeldige variablene kalles prediktorvariabler. I de fleste
tilfeller er ikke all variasjon relevant for regresjon, dvs. bare en
viss mengde variasjonen i prediktorene er relevante, og bare for en del
av variasjonen i responsen. Dette fører til en reduksjon av den lineære
regresjonsmodellen der man kan forestille seg et underrom av rommet som
spennesut av prediktorvariablene som inneholder all relevant informasjon
for et underrom av rommet spent ut av responsvariablene.

I denne avhandlingenprøver vi å sammenligne noen nye metoder som er
basert på Envelopemodellen og noen etablerte metoder som principal
komponent regresjon (PCR) og partiell minste kvadraters regresjon (PLS).
Sammenligningen tester disse metodene på deres ytelse til å produsere
minimum prediksjon- og estimeringsfeil, mens modelleringsdata simuleres
med spesielt designede egenskaper. For simuleringen har vi også laget en
R-pakke kalt \texttt{simrel} med et webgrensesnitt.

En simuleringsmodell for multirespons, multivariat lineær modell, som
simuleringsverktøyet bygger på, diskuteres i den første artikkelen.
Denne artikkelen utarbeider et grunnleggende fundament for simuleringene
basert på konseptet om reduksjon av regresjonsmodeller. Den andre
artikkelen diskuterer likhetene i Envelope-, PCR- og
PLS-populasjonsmodellene. Denne artikkelen sammenligner
prediksjonsytelsen til flere multivariate metoder ved bruk av en modell
med en enkelt respons.

De to siste artiklene gir en grundig evaluering av prediksjons- og
estimeringsegenskapene til etablerte metoder (PCR, PLS1 og PLS2) og
nyutviklede envelope-baserte metoder (Xenv og Senv). Ikke uventet fant
studien at det ikke finnes en enkelt metode som dominerer i alle
situasjoner, men resultatene deres avhenger av egenskapene til dataene
de modellerer. Imidlertid har envelope-baserte metoder vist
bemerkelsesverdig resultater i mange tilfeller, både når det gjelder
prediksjon og estimering. Studien anbefaler også forskere å bruke og
evaluere envelope-metodene.
