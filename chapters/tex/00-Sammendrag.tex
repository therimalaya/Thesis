En lineær regresjonsmodell definerer et lineært forhold mellom to eller
flere tilfeldige variabler. De tilfeldige variablene som er avhengige av
andre tilfeldige variabler kalles ofte responsvariabler og de uavhengige
tilfeldige variablene kalles prediktorvariabler. I de fleste tilfeller
er ikke alle variasjoner relevante for regresjon, dvs. bare en viss
mengde variasjoner i prediktorer er relevante for en del av variasjoner
i respons. Dette fører til en reduksjon av den lineære
regresjonsmodellen der man kan forestille seg et underområde av plassen
som spennes av prediktorvariablene som inneholder all relevant
informasjon for et underområde spandert av responsvariablene.

I denne avhandlingen prøver vi å sammenligne noen nye metoder som er
basert på konvoluttmodellen og noen etablerte metoder som
hovedkomponenter regresjon (PCR) og partiell minste kvadraters regresjon
(PLS). Sammenligningen tester disse metodene på deres ytelse til å
produsere minimum prediksjon og estimeringsfeil mens modelleringsdata
simuleres med spesielt designet egenskaper. For simuleringen har vi også
laget en R-pakke kalt \texttt{simrel} med et webgrensesnitt.

En første simuleringsmodell for en multirespons, multivariat lineær
modell som simuleringsverktøyet bygger på. Denne artikkelen utarbeider
et grunnleggende fundament for simuleringene med konseptet reduksjon av
regresjonsmodeller. Den andre artikkelen diskuterer likhetene i
konvolutt-, PCR- og PLS-populasjonsmodellene. Denne artikkelen
sammenligner prediksjonsytelsen til flere multivariate metoder ved bruk
av en modell med en enkelt respons.
